%-------------------------------------------------------------------------------
\section{Introduction}
%-------------------------------------------------------------------------------

Due to the success of fuzzing as a method of identifying bugs in software, many fuzzers have been 
developed. Unfortunately, selecting a fuzzer is a difficult task. The following challenges prevent 
researchers from quickly selecting a fuzzer using benchmarking: no fuzzer consistently outperforms 
the others, the efficacy of a fuzzer is inconsistent throughout its execution, and fuzzing results 
are not reproducible.

In response to these problems, \texttt{autofz} was developed. \texttt{Autofz} leverages multiple fuzzers by dynamically 
allocating resources to fuzzers using runtime data. This methodology empowers \texttt{autofz} to eliminate the 
fuzzer selection problem and outperform individual fuzzers.

In spite of these successes, \texttt{autofz} is not perfect. While there is no consensus on the best way to 
assess fuzzers, \texttt{autofz} oversimplifies this process. \texttt{Autofz} originally relied on a 
singular metric, AFL bitmap coverage, to rank its fuzzers and allocate resources. In this paper, we
propose the use of unique bugs as an additional and alternative metric. We posit that, in 
many cases, \texttt{autofz} performs even better if it considers the number of unique bugs found when assigning 
resources to individual fuzzers.

Moreover, \texttt{autofz} prides itself on streamlining the fuzzer selection process for the novel user, but 
it was designed for and tested on a cluster of hosts, each boasting an AMD Ryzen 9 3900 with 24 cores 
and 32 GB memory. We believe that this is not representative of the standard user, and as a result, 
autofz needs to be adapted for use in additional environments. \cite{fu_autofz_2023} We posit that \texttt{autofz} can 
effectively deliver similar results on commodity, portable hardware (i.e. Apple MacOS based laptop 
computers) that is based on the ARM64 architecture. 

By investigating the implementation and results of \texttt{autofz} on this configuration, we identify some the 
challenges and opportunities presented by fuzzing on the ARM64 architecture.

In this effort, we seek to improve \texttt{autofz} and answer the following research questions:
\begin{itemize}
    \item \textbf{RQ1.} Can we successfully reproduce a working tool chain from
    Fu et al. \cite{fu_autofz_2023} on portable ARM64 computing devices?
    \item \textbf{RQ2.} Can we reproduce the results (e.g. number of defects found)
     of fuzzing at least two targets from Fu et al. \cite{fu_autofz_2023} on our ARM64-based
      tool chain?
    \item \textbf{RQ3.} \texttt{autofz} uses a single metric (i.e. AFL bitmap coverage) for
     allocation of resources to fuzzers. Can we identify at least one additional,
     alternate metric that will improve this allocation?
\end{itemize}
