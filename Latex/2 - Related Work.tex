%-------------------------------------------------------------------------------
\section{Related Work}
%-------------------------------------------------------------------------------

\subsection{Fuzzing}
In 1990, Miller et al. proposed fuzz testing as a tool to identify bugs in Unix utilities and other 
applications. Their fuzzer generated, deployed, and monitored the execution of random strings on a 
target program. The application could succeed, crash, or hang. The application was successful when 
program terminated normally, and the input did not identify a bug. Alternatively, if the input caused 
the program to crash or hang, a defect was discovered. Using this simple methodology, Miller et al. 
were able to uncover bugs within Unix utilities that had not been previously identified using formal 
testing procedures. \cite{miller_empirical_1990}

Since Miller et al. introduced fuzzing as an effective means of identifying bugs, over thirty more 
sophisticated fuzzers have been proposed. Most fuzzers conform to the basic methodology introduced by 
Miller et al; they generate test cases and deploy them on a target while monitoring its state. 

Despite their similarities, the test case generation phase varies between fuzzers. Fuzzers can be 
divided as generation-based or mutation-based fuzzers. Generation-based fuzzers generate inputs that 
conform with grammars while mutation-based fuzzers modify existing inputs. 

Fuzzers can also be divided into blackbox, whitebox, and greybox. Blackbox fuzzers only monitor the 
state of the target application before and after the execution of an input while whitebox fuzzers 
document the status of the target throughout execution, and greybox fuzzers monitor specific 
execution attributes. \cite{zhu_fuzzing_2022}

These differences in implementation lead to variablitly in individuals fuzzers' fitness for
different targets.

\subsection{Fuzzing Assessment Methods}
Despite the abundance of fuzzers and their differences in implementation and performance, a consensus
 has not emerged on how to compare them.

\subsubsection{AFL Bitmap Coverage}
Bitmap coverage is a one means of capturing a fuzzer's performance; it quantifies the number of
 branches explored by the fuzzer. To calculate bitmap coverage, instrumentation is injected at branch 
 points when the program is compiled. By marking these points in the binary, then the fuzzer can 
 identify which branches have and have not yet been taken during a fuzzing run. 
 \cite{noauthor_afldocstechnical_detailstxt_nodate}

 This coverage method was introduced in the American Fuzzy Lop (AFL) project \cite{noauthor_afldocstechnical_detailstxt_nodate}. 
 It is supported by fuzzers that are based on or otherwise derived from AFL.  AFL's developers 
 note that this approach makes distinguishing between different program traces trivial. It also draws 
 the fuzzer to the target's state transition points, where opportunities to enter undocumented states 
 are most likely. The bitmap is implemented as 64 kB of shared memory, or enough storage to 
 accommodate 65,536 coverage points, \emph{l} \cite{yun_qsym_2018}, each roughly implemented as 
 described in Algorithm \ref{bitmap}

 \begin{algorithm}
    \caption {Insert Bitmap}\label{bitmap}
    \begin{algorithmic}[1]
        \Procedure {Insert Bitmap}{$location$}$\ as\ b(l)$
            \State $l_{cur} \gets\ COMPILE\_TIME\_RANDOM$
            \State $b(l) \gets\ b[l_{cur}\  xor\  l_{prev}] + 1$
            \State $l_{prev} \gets\ l_{cur} >> 1$
        \EndProcedure
    \end{algorithmic}
\end{algorithm}
 
While using bitmap coverage allows competition amongst disparate fuzzers, this choice restricts the 
set of Autofz's supported fuzzers to those that implement AFL bitmap as a coverage metric. In algorithm 
\ref{bitmap}\, we define \emph{$l_{cur}$} as a random number that is assigned at compile time. While 
not implemented in our experiments, the bitmap metric can be implemented when testing black-box binaries; 
i.e. binary programs for which source code is not available to the tester. AFL (and some of its derivatives)
implements QEMU mode instrumentation \cite{noauthor_aflqemu_mode_nodate}, where bitmap composition is 
by \cite{noauthor_afldocstechnical_detailstxt_nodate} is described in Algorithm \ref{qemu_bitmap}.

\begin{algorithm}
    \caption {Insert Bitmap, QEMU Mode}\label{qemu_bitmap}
    \begin{algorithmic}[1]
        \Procedure {Insert Bitmap}{$location$}$\ as\ b(l)$
          \If{$end_{elf_text} > addr_{block} > start_{elf_text}$}
                \State $l_{cur} \gets\ (addr_{block}\ >>\ 4)\ xor\ (addr_{block}\ <<\ 8)$
                \State $b(l) \gets\ b[l_{cur}\  xor\  l_{prev}] + 1$
                \State $l_{prev} \gets\ l_{cur} >> 1$
          \EndIf
        \EndProcedure
    \end{algorithmic}
\end{algorithm}

\subsubsection{Additional Metrics}
Ecezia et al. \cite{eceiza_improving_2023} asserted that the methodology for testing new and 
evaluating existing fuzzers needs to be standardized. In their proposed protocol, they 
recommended using three categories of metrics. They are bugs, coverage, and performance. 

\begin{enumerate}
    \item \textbf{Bugs} Ecezia et al. recommends using the total number of
    bugs discovered as the best means of assessing the general performance 
    of a fuzzer. They argue that since number of bugs discovered 
    encompasses all other bug metrics, it is the superior bug metrics. 
    While total number of bugs identified may encompass unique bugs 
    discovered, unique bugs discovered is likely a better means of 
    measuring a fuzzers general performance. Total number of bugs 
    discovered includes duplicate bugs. This may result in a fuzzer that 
    discovers the few bugs many times appearing to perform better than 
    another fuzzer that discovers more unique bugs once. Moreover, once a 
    bug is known, it does not need to be discovered again.
    \item \textbf{Coverage} Coverage metrics communicate the performance 
    of a fuzzer's exploratory processes. While Ecezia et al. do not 
    explicitly identify bitmap coverage in their proposed method, they 
    recommend a combination of line and branch coverage for evaluating the 
    coverage achieved by a fuzzer. Despite this recommendation, they 
    maintain that complete line or branch coverage does not guarantee 
    complete bug discovery. Instead, they report that if the fuzzer does 
    not provide the target bug triggering input, an error may remain 
    undiscovered even after the buggy line of or path through code is 
    fuzzed. 
    \item \textbf{Performance} Performance metrics reflect the efficiency 
    of the fuzzer. Ecezia et al. reports that density is the best means of 
    measuring a fuzzer's efficiency. Density communicates the ratio of 
    number of tests completed to bugs discovered. A denser fuzzer will 
    need few runs to identify a bug, and consequently, identify more bugs 
    faster than a fuzzer with a lower density. 
\end{enumerate}

\subsubsection{Experimental Conditions}
In addition to establishing the evaluation metrics, Ecezia et al. established  criteria for fuzzer 
test conditions. They asserted that in order to compare fuzzers, the fuzzers must be executed 
against the same applications, at least 15 times, and for the same period of time. These conditions 
are aimed at eliminating inconsistencies associated with evaluating fuzzers. 

For instance, fuzzers can only be compared if they are run on the same target application because 
the performance of a fuzzer is directly linked to the system under test. When a more buggy program 
is fuzzed, it will yield more bugs than a well-designed system regardless of the fuzzer. 

Moreover, when evaluating a fuzzer, it must be executed at least 15 times because the random input 
generated by a fuzzers will vary between runs.

The final condition that must be constant for each test is execution time; a fuzzer executed for 
10 hours cannot be compared to a fuzzer that has been running for 24 hours; the fuzzer that ran for 
24 hours will likely discover more bugs than the fuzzer that has run for 10 hours. The 24 hour fuzzer 
is not necessarily a better fuzzer, but it had more opportunities to discover unwanted behavior. 
\cite{eceiza_improving_2023}

\subsection{Autofz}
While Ecezia et al. looked to simplify the difficult task of identifying the best fuzzer by establishing 
evaluation criteria, Fu et al.\cite{fu_autofz_2023} aimed to eliminate the task. They proposed a new 
fuzzer that would make the task of evaluating and selecting an individual fuzzer obsolete. Autofz is a 
dynamic fuzzer that dynamically deploys a set of fuzzers. Autofz accomplishes this by dividing its 
workload into two phases; they are a preparation and a focus phase.

During the preparation phase, Autofz captures the runtime trends of multiple fuzzers on the target 
application. First, Autofz deploys the fuzzers with the same seeds. Each fuzzer executes for an equal 
portion of the preparation phase. Next, Autofz measures bitmap coverage achieved by each individual fuzzer. 
This process is repeated until the end of the preparation phase is reached or a strong trend emerges. A 
strong trend occurs when the difference in bitmap coverage of the best and worst performing fuzzer exceeds 
a predetermined threshold. 

Based on the data collected in the preparation phase, Autofz deploys the set of fuzzers with the potential 
to maximize the focus phase’s performance. More specifically, Autofz allocates resources to the fuzzers 
proportionally to their performance during the preparation phase. For example, the fuzzers that demonstrated 
the greatest bitmap coverage during the preparation phase have the most resources allocated to them during 
the focus phase, and the fuzzers that preformed poorly will have less available. This may result in one 
fuzzer receiving all available resources or no resources. After assigning resources to the individual fuzzers, 
the focus phase is executed. In the focus phase, each fuzzer is executed with its respective resources.

During the focus phase, Autofz achieved greater bitmap coverage than individual fuzzers that executed under 
the same conditions. \cite{fu_autofz_2023}
