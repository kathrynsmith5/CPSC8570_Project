%-------------------------------------------------------------------------------
\section{Methodology}
%-------------------------------------------------------------------------------
In order to confirm the findings presented in Fu et.al.\cite{fu_autofz_2023}, we implemented our 
own instances of autofz. We implemented our evaluation environment by:

\begin{enumerate}
    \item Installing the containers provided by Fu et al. \cite{fu_autofz_2023} on our own
    AMD64 development workstation;
    \item Building our own containers, based on Fu et al. \cite{fu_autofz_2023} on our own
    AMD64 development workstation;
    \item Building our own containers, based on our own modifications to Autofz on our own 
    ARM64 (Apple MacBook) laptop devices
\end{enumerate}

Moreover, we tested our implementations of autofz against the targets: exiv2, mp3gain, mojs, and 
tcpdump. Our initial findings validate the results presented in Fu et al.; autofz 
is consistently a top performing against the tested targets.

%-------------------------------------------------------------------------------
\subsection{Autofz on AMD64 - Validating Fu et al.\cite{fu_autofz_2023}}
%-------------------------------------------------------------------------------

To validate the results of Fu et al., we deployed a single physical workstation with 
an AMD FX\texttrademark - 6300 Six-Core Processor (3.5 Ghz), 16 GB memory, and 1 TB 
SSD, running Ubuntu 20.04.6 LTS (focal). Containers were built on this host using 
the source code available at the sslab-gatech/autofz repository \cite{noauthor_sslab-gatechautofz_2024} 
and were also installed from fuyu0425/autofz images on docker hub\cite{noauthor_fuyu0425autofz_nodate}.


%-------------------------------------------------------------------------------
\subsection{Autofz on ARM64 - ARMy Fuzzing}
%-------------------------------------------------------------------------------

We have also been able to produce a working tool chain on ARM64, with some significant 
variations from Fu et al. \cite{fu_autofz_2023}. Our ARM64 environment supports 7 of 
the original fuzzers in Ubuntu 22.04 LTS. We configured these in virtual machines 
with the UTM app on host laptops running macOS 14.3.1 on Apple Silicon (ARM64) 
architecture \textbf{(RQ1)}. Key differences are described in Table \ref{arm64-characteristics}.

\begin{table}[ht]
    \begin{tabular}{|l|l|l|l|}
        \hline
                        & Original\cite{fu_autofz_2023} & ARM64 & Reason \\
        \hline
        Ubuntu Version  & 16.04             & 22.04 & 1 \\
        \hline
        aflforkserver.so    & x86\_64           & aarch64 & 2 \\
        \hline
        AFL (original)  & Yes               & Yes & \\
        \hline
        AFLFast         & Yes               & Yes & \\
        \hline
        Angora          & Yes               & No & 4 \\
        \hline
        Fairfuzz        & Yes               & Yes & \\
        \hline
        LAF-Intel       & Yes               & Yes & \\
        \hline
        LearnAFL        & Yes               & No & 3 \\
        \hline
        LibFuzzer       & Yes               & Yes & \\
        \hline
        QSYM            & Yes               & No & 4 \\
        \hline
        Radamsa         & Yes               & Yes & \\
        \hline
        RedQueen        & Yes               & Yes & \\
        \hline
    \end{tabular}
    \caption{Key differences between Original AMD64 and ARM64 Environments}
    \label{arm64-characteristics}
\end{table}

The majority of our changes were made in the \texttt{build.sh} scripts and \texttt{Dockerfile}s
for the affected containers.

There are a number of reasons for the differences between Fu et al.,
and our ARM64 environment. They correspond to the reason numbers in Table 
\ref{arm64-characteristics} and are:

\begin{enumerate}
    \item Ubuntu 16.04 LTS does not support the ARM64 architecture (also called 
    aarch64).
    \item Aflforkserver.so comes from the quickcov package, developed to support 
    the Cupid research project\cite{guler_cupid_2020}. We had to recompile it for aarch64.
    \item Learn AFL does not properly compile instrumented binaries for aarch64.
    \item Angora and QSYM are written such that they are dependent upon the x86\_64 ISA.
    They do not support other architectures.
\end{enumerate}