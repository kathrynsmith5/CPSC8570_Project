%-------------------------------------------------------------------------------
\section{Future Research}
%-------------------------------------------------------------------------------
\subsection{Additional Targets} The version of \texttt{autofz} proposed by Fu et al. 
was tested against 12 targets and verified over 10 iterations. Our research  
confirmed the finding of Fu et al on 4 targets over ten repetitions. While we 
replicated the outcomes of Fu et al. with limited tests, the variations 
of \texttt{autofz} proposed in this paper need to under go additional testing. Due to their 
novelty, they cannot rely on previous literature to confirm their findings.

Similarly to \texttt{autofz}, we tested \texttt{autofz} for ARM64 against 4 targets over 10 
iterations. While trends were able to be identity among the tested targets, this
sample is not large enough to apply the outcomes to all targets. 

Additionally, the proposed algorithms for resources allocation during the preparation
underwent restricted testing, and they were insconsistant between targets. 
More testing needs to be undertaken in order to confirm the performance on tested targets and
extrapolate their performance to untested targets.

\subsection{Complex Targets} All versions of \texttt{autofz} need to be tested against 
complex targets. Fu et al. evaluated \texttt{autofz} against simple targets with known bugs. In 
order to compare our findings to Fu et al., we were restricted to testing our 
iterations of \texttt{autofz} against the same targets. In the future, all variations of 
\texttt{autofz} should be tested against larger, more complex targets that are comparable to
targets that may leverage fuzzing.

\subsection{Fuzzer Pool Composition} Fu et al. disclosed that the performance of
Autofz is dependent of the composition of its fuzzer pool. When \texttt{autofz} was presented
with a large pool of fuzzers and when \texttt{autofz} was provided with too few fuzzers, it 
under preformed. Further testing needs to be undertaken in order to determine the ideal
number of fuzzer in the pool as well as which fuzzers should be in the pool.

\subsection{Documentation} In addition to confirming and maximizing the performace of \texttt{autofz},
\texttt{autofz} needs undergo additional documentation. In route of evaluating \texttt{autofz}, significant 
changes were made to its graphing function. While these simplified comparing the performance of the variations
of \texttt{autofz}, they did not improve the performance of \texttt{autofz} and were outside of the scope of our research.
As a result, these modifications were not documented and need to testing. 



