%-------------------------------------------------------------------------------
\section{Future Research and Limitations}
%-------------------------------------------------------------------------------
\item \textbf{Additional Targets.} The version of Autofz proposed by Fu et al. 
was tested against 12 targets and verified over 10 iterations. Our research  
confirmered the finding of Fu et al on 4 targets over ten repetitions. While we 
replicated the outcomes of Fu et al. with limited tests, the variations 
of Autofz proposed in this paper need to under go additional testing. Due to their 
novelty, they cannot rely on previous literature to confirm their findings.

Similarly to Autofz, we tested Autofz for ARM64 against 4 targets over 10 
iterations. While trends were able to be identied among the tested targets, this
sample is not large enough to apply the outcomes to all targets. 

Additionally, the proposed algorithms for resources alloaction during the preparation
underwent restricted testing. More testing needs to be undertaken in order to confirm
the preformance on tested targets and extrapolate the performance to untested targets.

\item \textbf{Complex Targets.} All versions of Autofz need to be tested against 
complex targets. Fu et al. evaluated Autofz against simple targets with known bugs. In 
order to compare our findings to Fu et al., we were restiricted to testing our 
iterations of Autofz against the same targets. In the future, all veriations of 
Autofz should be tested agaist larger, more complex targets that are comparable to
targets that may leverage fuzzing.

\item \textbf{Fuzzer Pool Composition} Fu et al. disclosed that the performance of
Autofz is dependent of the composision of its fuzzer pool. When Autofz was presented
with a large pool of fuzzers and when Autofz was provided with too few fuzzers, it 
underpreformed. Further testing needs to be undertaken in order to determine the ideal
number of fuzzer in the pool as well as which fuzzers should be in the pool.