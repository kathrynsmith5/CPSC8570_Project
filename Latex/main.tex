%%%%%%%%%%%%%%%%%%%%%%%%%%%%%%%%%%%%%%%%%%%%%%%%%%%%%%%%%%%%%%%%%%%%%%%%%%%%%%%%
% Template for USENIX papers.
%
% History:
%
% - TEMPLATE for Usenix papers, specifically to meet requirements of
%   USENIX '05. originally a template for producing IEEE-format
%   articles using LaTeX. written by Matthew Ward, CS Department,
%   Worcester Polytechnic Institute. adapted by David Beazley for his
%   excellent SWIG paper in Proceedings, Tcl 96. turned into a
%   smartass generic template by De Clarke, with thanks to both the
%   above pioneers. Use at your own risk. Complaints to /dev/null.
%   Make it two column with no page numbering, default is 10 point.
%
% - Munged by Fred Douglis <douglis@research.att.com> 10/97 to
%   separate the .sty file from the LaTeX source template, so that
%   people can more easily include the .sty file into an existing
%   document. Also changed to more closely follow the style guidelines
%   as represented by the Word sample file.
%
% - Note that since 2010, USENIX does not require endnotes. If you
%   want foot of page notes, don't include the endnotes package in the
%   usepackage command, below.
% - This version uses the latex2e styles, not the very ancient 2.09
%   stuff.
%
% - Updated July 2018: Text block size changed from 6.5" to 7"
%
% - Updated Dec 2018 for ATC'19:
%
%   * Revised text to pass HotCRP's auto-formatting check, with
%     hotcrp.settings.submission_form.body_font_size=10pt, and
%     hotcrp.settings.submission_form.line_height=12pt
%
%   * Switched from \endnote-s to \footnote-s to match Usenix's policy.
%
%   * \section* => \begin{abstract} ... \end{abstract}
%
%   * Make template self-contained in terms of bibtex entires, to allow
%     this file to be compiled. (And changing refs style to 'plain'.)
%
%   * Make template self-contained in terms of figures, to
%     allow this file to be compiled. 
%
%   * Added packages for hyperref, embedding fonts, and improving
%     appearance.
%   
%   * Removed outdated text.
%
%%%%%%%%%%%%%%%%%%%%%%%%%%%%%%%%%%%%%%%%%%%%%%%%%%%%%%%%%%%%%%%%%%%%%%%%%%%%%%%%

\documentclass[letterpaper,twocolumn,10pt]{article}
\usepackage{usenix-2020-09}

% to be able to draw some self-contained figs
\usepackage{tikz}
\usepackage{amsmath}
\usepackage{comment}

%-------------------------------------------------------------------------------
\begin{document}
\graphicspath{ {./images/} }
%-------------------------------------------------------------------------------

%don't want date printed
\date{}

% make title bold and 14 pt font (Latex default is non-bold, 16 pt)
\title{\Large \bf “ARMy Fuzzing:” Midterm Report}

%for single author (just remove % characters)
\author{
{\rm Andrew York}\\
Clemson University
\and
{\rm Kathryn Smith}\\
Clemson University
} % end author

\maketitle

%-------------------------------------------------------------------------------
\section{Introduction and Problem}
%-------------------------------------------------------------------------------

At the midterm point in this term, this paper reports our progress thus far on 
this project. We've built a version of the autofz tool chain on ARM64. We have 
also successfully executed fuzzing campaigns against four targets that were included 
in the original autofz paper. We look forward to selecting and integrating additional 
metrics into autofz that can be used to improve the allocation of resources to the 
individual algorithms used by autofz. This paper also documents each researcher's 
individual completed and planned contributions for this project.

Due to the success of fuzzing as a method of identifying bugs in software, many fuzzers have been developed. 
Unfortunately, selecting a fuzzer is a difficult task. The following challenges 
prevent researchers from quickly selecting a fuzzer using benchmarking: no fuzzer 
consistently outperforms the others, the efficacy of a fuzzer is inconsistent 
throughout its execution, and fuzzing results are not reproducible. 

In response to these problems, Autofz was developed. Autofz leverages multiple 
fuzzers by dynamically allocating resources to the fuzzers using runtime data. 
Autofz eliminates the fuzzer selection problem and consistently outperforms 
individual fuzzers. 

In spite of its successes, Autofz is not perfect. While there is no consensus on
the best way to assess fuzzers, Autofz oversimplifies this process. Autofz relies
on one metric, path coverage, to rank its fuzzers and allocate resources. Autofz 
could be improved by incorporating additional metrics during its analysis. 

Moreover, Autofz prides itself on streamlining the fuzzer selection process for 
the novel user, but it was designed and tested on Ubuntu 20.04 equipped with AMD 
Ryzen 9 3900 having 24 cores and 32 GB memory. This equipment is not representative 
of the standard user, and as a result, Autofz needs to be rebuilt and tested in 
additional environments. \cite{Fu}

%-------------------------------------------------------------------------------
\section{Related Work}
%-------------------------------------------------------------------------------

\subsection{Fuzzing}
In 1990, Miller et al. proposed fuzz testing as a tool to test Unix utilities 
and other applications. Their fuzzer generated random number and character 
strings. Next, they executed a target program with the random input. Finally, 
the fuzzer would record how the tested application responded to the random input. 
The application could succeed, crash, or hang; succeed meant that the program 
executed normally, a crash indicated that the program terminated abnormally while 
a hang implied that the application entered an infinite loop. Furthermore, a 
crash or hang indicated unwanted behavior within the target application. Using this 
simple methodology, Miller et al. discovered a wealth of bugs within Unix utilities 
that had not been identified using formal testing procedures. \cite{Miller}

\subsection{Fuzzing Assessment Methods}
Since Miller et al. introduced fuzzing as an inexpensive means of identifying bugs 
and increasing overall system reliability, many researchers have developed more 
complex fuzzers. Despite this abundance of fuzzers, a consensus has not emerged 
on how to compare them. Moreover, testing conditions are not consistent for new fuzzers. 
Ecezia et al. aimed to resolve this issue by proposing a universal method for testing new 
fuzzers and evaluating existing fuzzers. They presented evaluation metrics that could be 
divided into three categories: bug detection, coverage, and performance. Metrics defined 
in the bug detention group quantified any undesirable behavior identified by the fuzzer 
while metrics categorized under coverage measured the percentage of the code that had 
been executed by the fuzzer. Metrics in the performance category are measurements not 
directly related to bug detection or performance; they include data such as number of 
runs executed during a set time frame and time needed to identify the first bug.

After establishing the metrics that need to be collected during testing, Ecezia et al. 
set the criteria for fuzzer test conditions. They asserted that in order to compare 
multiple fuzzers, the fuzzers must be tested against the same target applications, at 
least 15 times, and for the same period of time. These conditions are aimed at eliminating 
inconsistencies associated with evaluating fuzzers. For instance, fuzzers can only be 
compared if they are run on the same target application because the results of fuzzers 
are directly linked to the system under test. When a more buggy program is fuzzed, it 
will yield more bugs than a well-designed system regardless of the fuzzer. Moreover, 
in order to evaluate a fuzzer, it must be executed at least 15 times because the random 
input generated by a fuzzers will vary between runs. The final condition that must be 
constant for each test is execution time; a fuzzer executed for 10 hours cannot be 
compared to a fuzzer that has been running for 24 hours; the fuzzer run for 24 hours 
will likely discover more bugs than the fuzzer that has run for 10 hours. The 24 hour 
fuzzer is not necessarily a better fuzzer, but it had more opportunities to discover 
unwanted behavior. \cite{Ecezia}

\subsection{Autofz}
While Ecezia et al. looked to simplify the difficult task of identifying the best 
fuzzer by establishing evaluation criteria, Fu et al. aimed to eliminate the task. 
They proposed a new fuzzer that would make the task of evaluating and selecting an 
individual fuzzer obsolete. Autofz is a collaborative fuzzer that dynamically deploys 
a set of fuzzers. Autofz accomplishes this by dividing its workload into 2 phases; 
they are a preparation and a focus phase. 

During the preparation phase, Autofz captures the runtime trends of multiple fuzzers
on the target application. First, Autofz deploys the fuzzers with the same seeds. 
These fuzzers are allowed to execute for the same period of time. Next, Autofz 
measures the number of unique paths explored by the individual fuzzers. This 
process is repeated until the phase times out or a strong trend emerges. A strong 
trend occurs when the difference between the code coverage of the best and worst 
performing fuzzer exceeds a predetermined threshold. 

Based on the data collected in the preparation phase, Autofz deploys 
the set fuzzers with the potential to maximize the focus phase’s performance.
Specifically, Autofz allocates resources to the fuzzers proportionally 
to their performance during the preparation phase. For example, the fuzzers
that demonstrated the greatest code coverage during the preparation phase 
have the most resources allocated to them during the focus phase. Moreover, 
one fuzzer can be allocated all available resources or no resources. After 
assigning resources to the individual fuzzers, the focus phase is executed. In 
the focus phase, each fuzzer is executed with its respective resources.

During the focus phase, Autofz achieved greater bitmap coverage than individual 
fuzzers that executed under the same conditions. Autofz also outperformed 
collaborative fuzzers that equally allocated resources to individual fuzzers. 
\cite{Fu}

%-------------------------------------------------------------------------------
\section{Remaining Plan, Timeline}
%-------------------------------------------------------------------------------
In this effort, we seek to answer the following research questions:

\begin{itemize}
    \item \textbf{RQ1.} Can we successfully reproduce a working tool chain from 
    Fu et al. \cite{Fu} on portable ARM64 computing devices?
    \item \textbf{RQ2.} Can we reproduce the results (e.g. number of defects found)
     of fuzzing at least two targets from Fu et al. \cite{Fu} on our ARM64-based
      tool chain?
    \item \textbf{RQ3.} Autofz uses a single metric (i.e. AFL bitmap coverage) for
     allocation of resources to fuzzers. Can we identify at least one additional, 
     alternate metric that will improve this allocation?
\end{itemize}

Our plan for the remaining time in this term is as follows:

\begin{enumerate}
    \item In \textbf{March}, we will implement at least one metric proposed in February
     and compare its effectiveness against the existing version of Autofz (i.e. AFL bitmap).

    \item In \textbf{April}, we will compile our results and document our findings.
\end{enumerate}
%-------------------------------------------------------------------------------

%-------------------------------------------------------------------------------
\section{Progress to Date}
%-------------------------------------------------------------------------------
In order to confirm the findings presented in Fu et.al.\cite{Fu}, we deployed our own instance of 
their autofz environment. We replicated their testing environment by deploying their docker image 
on our own hardware. Moreover, we tested our recreation of Fu et al.'s autofz against the targets: 
exiv2, nm, mojs, and tcpdump. Our initial findings validate the results presented in Fu et al.; 
autofz is consistently a top preforming fuzzers against the tested targets. 

We have also been able to produce a working tool chain on ARM64, with some significant variations
from Fu et al. \cite{Fu}. Our ARM64 environment implements 7 fuzzing algorithms in
Ubuntu 22.04 LTS, running in a virtual machine with the UTM app on host laptops running
macOS 14.3.1 on Apple Silicon (ARM64) architecture \textbf{(RQ1)}. Key differences are
described in Table \ref{arm64-characteristics}.

\begin{table}[ht]
    \begin{tabular}{|l|l|l|l|}
        \hline
                        & Original\cite{Fu} & ARM64 & Reason \\
        \hline
        Ubuntu Version  & 16.04             & 22.04 & 1 \\
        \hline
        aflforkserver.so    & x86\_64           & aarch64 & 2 \\
        \hline
        AFL (original)  & Yes               & No & 3 \\
        \hline
        AFLFast         & Yes               & Yes & \\
        \hline
        Angora          & Yes               & No & 4 \\
        \hline
        Fairfuzz        & Yes               & Yes & \\
        \hline
        LAF-Intel       & Yes               & Yes & \\
        \hline
        LearnAFL        & Yes               & Yes & \\
        \hline
        LibFuzzer       & Yes               & Yes & \\
        \hline
        QSYM            & Yes               & No & 4 \\
        \hline
        Radamsa         & Yes               & Yes & \\
        \hline
        RedQueen        & Yes               & Yes & \\
        \hline
    \end{tabular}
    \caption{Key differences between Original AMD64 and ARM64 Environments}
    \label{arm64-characteristics}
\end{table}

There are a number of reasons for the differences between Fu et al., 
and our ARM64 environment. They correspond to the reason numbers in Table \ref{arm64-characteristics}
and are:
\begin{enumerate}
    \item Ubuntu 16.04 LTS does not support the ARM64 architecture (also called aarch64).
    \item Aflforkserver.so comes from the quickcov package, developed to support the Cupid
    research project\cite{guler2020cupid}. We had to recompile it for aarch64.
    \item Original AFL does not compile instrumented binaries for aarch64. Instead, we
    implement AFL++.
    \item Angora and QSYM are written such that they are dependent upon the x86\_64 ISA.
    They do not support other architectures.
\end{enumerate}

To compare the  modified autofz with the proposed autofz, we tested it against the targets: exiv2, 
nm, mojs, and tcpdump. Initial test indicate that the ARM64 compatible autofz performs similarly to 
the unmodified autofz. Figure \ref{fig:exiv2_bmdensity_8Mar2024} displays bitmap density covered of 
the individual algorithms in our implementation, as compared to a similar plot from Fu et. al\cite{Fu}.

\begin{figure}
    \includegraphics[width=0.52\textwidth]{exiv2_bmdensity_8Mar2024.png}
    \centering
    \caption{A comparison of bitmap density covered in the original\cite{Fu} and our coverage during
     initial fuzzing of exiv2}
    \label{fig:exiv2_bmdensity_8Mar2024}
\end{figure}

Bitmap density coverage of our initial fuzzing campaign against tcpdump is plotted in figure 
\ref{fig:tcpdump_bmdensity_8Mar2024}, shown as a comparison with results from Fu et al.\cite{Fu}.


\begin{figure}
    \includegraphics[width=0.52\textwidth]{tcpdump_bmdensity_8Mar2024.png}
    \centering
    \caption{A comparison of bitmap density covered in the original\cite{Fu} and our coverage during initial fuzzing of tcpdump}
    \label{fig:tcpdump_bmdensity_8Mar2024}
\end{figure}

In addition to producing an ARM64 compatible autofz, we identified a potential improvement for autofz. 
Existing versions of autofz use bitmap coverage to rank fuzzers during the preparation, but bitmap 
coverage may not be the best metric for identifying the most effective fuzzers for a target. While 
bitmap coverage indicates the portion of a target that was fuzzed, it does not guarantee bug discovery. 
To resolve these problems, we plan to modify autofz, so fuzzers that discover more bugs during the 
preparation phase are rewarded with more resources during the focus phase because discovering more bugs 
across a smaller area of code decreases the attack surface by more entry points than discovering fewer 
bugs over a greater area of the code.

%-------------------------------------------------------------------------------
\section{Team Contributions and Plan}
%-------------------------------------------------------------------------------

At the midpoint in this term, Kathryn has recreated autofz using Fu et al.\cite{Fu}'s build scripts 
on our AMD64 development system. She identified an alternate metric that we can incorporate into 
autofz. She also executed fuzzing campaigns against mojs and tcpdump on both of our architectures 
under evaluation (i.e. AMD64 and ARM64). 

Andrew recreated autofz using Fu et al.\cite{Fu}'s docker image on our AMD64 development system. 
He also completed the modifications necessary to run autofz on our ARM64 laptops. He executed fuzzing 
campaigns against exiv2 and nm on both: AMD64 and ARM64.

We will both work to reproduce and confirm results from Fu et al.\cite{Fu}. To validate our findings: 
Andrew plans to analyze his findings from fuzzing runs against exiv2 and nm; Kathryn plans to do the 
same of findings from fuzzing runs against mojs and tcpdump. We will work together to incorporate 
the rate of unique bugs found as an additional metric into the allocation of resources to individual 
algorithms by autofz.



    
\bibliographystyle{plain}
\bibliography{output}

%%%%%%%%%%%%%%%%%%%%%%%%%%%%%%%%%%%%%%%%%%%%%%%%%%%%%%%%%%%%%%%%%%%%%%%%%%%%%%%%
\end{document}
%%%%%%%%%%%%%%%%%%%%%%%%%%%%%%%%%%%%%%%%%%%%%%%%%%%%%%%%%%%%%%%%%%%%%%%%%%%%%%%%

%%  LocalWords:  endnotes includegraphics fread ptr nobj noindent
%%  LocalWords:  pdflatex acks
